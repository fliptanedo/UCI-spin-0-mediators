%   This file is based on the the APS files in the REVTeX 4 distribution. Version 4.0 of REVTeX, August 2001, Copyright (c) 2001 The American Physical Society.
%
% TeX'ing this file requires that you have AMS-LaTeX 2.0 installed as well as the rest of the prerequisites for REVTeX 4.0

\documentclass[showpacs,preprintnumbers,amsmath,amssymb]{revtex4}

\usepackage{amsmath}
\usepackage{amssymb}

%%%%%%%%%%%%%%%%%%%%%%%%%%%%%%%%%%%%%%%%%%%%%%%%%%%%%%%%%%%%%%%%%%
\font\fiverm=cmr5
%   the stuff below defines \eqalign and \eqalignno in such a
%   way that they will run on Latex
\newskip\humongous \humongous=0pt plus 1000pt minus 1000pt
\def\caja{\mathsurround=0pt}
\def\eqalign#1{\,\vcenter{\openup1\jot \caja
    \ialign{\strut \hfil$\displaystyle{##}$&$
    \displaystyle{{}##}$\hfil\crcr#1\crcr}}\,}
\newif\ifdtup
\def\panorama{\global\dtuptrue \openup1\jot \caja
    \everycr{\noalign{\ifdtup \global\dtupfalse
    \vskip-\lineskiplimit \vskip\normallineskiplimit
    \else \penalty\interdisplaylinepenalty \fi}}}
\def\eqalignno#1{\panorama \tabskip=\humongous
    \halign to\displaywidth{\hfil$\displaystyle{##}$
    \tabskip=0pt&$\displaystyle{{}##}$\hfil
    \tabskip=\humongous&\llap{$##$}\tabskip=0pt
    \crcr#1\crcr}}
%   The oldref and fig macros are for formatting
%   references and figure lists at the end of the paper.
%   If you type \oldref{1}Dirac, P.A.M. you will get
%   [1] Dirac, P.A.M.
%   Same goes for \fig except you get Figure 2.1
\def\oldrefledge{\hangindent3\parindent}
\def\oldreffmt#1{\rlap{[#1]} \hbox to 2\parindent{}}
\def\oldref#1{\par\noindent\oldrefledge \oldreffmt{#1}
    \ignorespaces}
\def\figledge{\hangindent=1.25in}
\def\figfmt#1{\rlap{Figure {#1}} \hbox to 1in{}}
\def\fig#1{\par\noindent\figledge \figfmt{#1}
    \ignorespaces}
%
%   This defines et al., i.e., e.g., cf., etc.
\def\ie{\hbox{\it i.e.}{}}  \def\etc{\hbox{\it etc.}{}}
\def\eg{\hbox{\it e.g.}{}}  \def\cf{\hbox{\it cf.}{}}
\def\etal{\hbox{\it et al.}}
\def\dash{\hbox{---}}
%   common physics symbols
\def\tr{\mathop{\rm tr}}
\def\Tr{\mathop{\rm Tr}}
\def\Im{\mathop{\rm Im}}
\def\Re{\mathop{\rm Re}}
\def\bR{\mathop{\bf R}{}}
\def\bC{\mathop{\bf C}{}}
%\def\Lie{\mathop{\cal L}}  % fancy L for the Lie derivative
\def\partder#1#2{{\partial #1\over\partial #2}}
\def\secder#1#2#3{{\partial^2 #1\over\partial #2 \partial #3}}
\def\bra#1{\left\langle #1\right|}
\def\ket#1{\left| #1\right\rangle}
\def\VEV#1{\left\langle #1\right\rangle}
\def\gdot#1{\rlap{$#1$}/}
\def\abs#1{\left| #1\right|}
\def\pr#1{#1^\prime}
\def\ltap{\raisebox{-.4ex}{\rlap{$\sim$}} \raisebox{.4ex}{$<$}}
\def\gtap{\raisebox{-.4ex}{\rlap{$\sim$}} \raisebox{.4ex}{$>$}}
% \contract is a differential geometry contraction sign _|
\def\contract{\makebox[1.2em][c]{
    \mbox{\rule{.6em}{.01truein}\rule{.01truein}{.6em}}}}
\def\slash#1{#1\!\!\!/\!\,\,}
\def\beq{\begin{equation}}
\def\eeq{\end{equation}}
\def\bea{\begin{eqnarray}}
\def\eea{\end{eqnarray}}
\def\half{\frac{1}{2}}
\def\aeq{\eeq}
\def\bq{\begin{quote}}
\def\eq{\end{quote}}
\def \Msol {M_\odot}
\def\GeV{\,{\rm GeV}}
\def\eV {\,{\rm  eV}}
\def\Mpc{\,{\rm Mpc}}
\def\pc{\,{\rm pc}}
\def\half{\frac{1}{2}}
%% macros to produce the symbols "less than or of order of"
%% and "greater than or of order of" %
\def \lta {\mathrel{\vcenter
     {\hbox{$<$}\nointerlineskip\hbox{$\sim$}}}}
\def \gta {\mathrel{\vcenter
     {\hbox{$>$}\nointerlineskip\hbox{$\sim$}}}}
%% a few convenient (?) abbreviations: %
\def \endpage {\vfill \eject}
\def \endline {\hfill \break}
\def \etal {{\it et al.}\ }
\relax

%\input math_macros
%\input prepictex
%\input pictex
%\input postpictex
\newdimen\tdim
\tdim=\unitlength
\def\bar{\overline}
\def\stpltsmbl{\setplotsymbol ({\small .})}
\def\tarrow{\arrow <8\tdim> [.3,.6]}
\def\moose#1#2#3{\tarrow from #1 to #2 \plot #2 #3 /}
\def\mooseb#1#2#3{\tarrow from #3 to #2 \plot #2 #1 /}
%%%%%%%%%%%%%%%%%%%%%%%%%%%%%%%%%%%%%%%%%%%%%%%%%%%%%%%%%%%%%%%%%%

\textheight=220mm \textwidth=165mm \topmargin=-10mm
\oddsidemargin=-1mm


\usepackage[
%	pdftitle={},
%	pdfauthor={},
	colorlinks=true,
	citecolor=black,
	linkcolor=black,
	urlcolor=blue,
	hypertexnames=false]{hyperref}


\begin{document}
\setcounter{page}{1}
\pagestyle{plain}

\Large

\begin{center}
{\bf MadGraph 5 Model: Spin-0 Mediators}
\end{center}

\begin{center}
Anthony DiFranzo, \,
Philip Tanedo, \,
Tim M.\ P.\ Tait

\href{mailto:adifranz@uci.edu}{adifranz@uci.edu}, \,
\href{mailto:flip.tanedo@uci.edu}{flip.tanedo@uci.edu}, \,
\href{mailto:ttait@uci.edu}{ttait@uci.edu}, \,

{\it Department of Physics \& Astronomy, University of California, Irvine, \textsc{ca} 92697}
\end{center}


\section{Purpose}

These two MadGraph 5 (MG5) \cite{Alwall:2011uj} model files are intended to simulate the production of Dirac fermion dark matter at a collider through a spin-0 mediator. The two models are called:
\begin{itemize}
\item {\tt GC\_scalar\_UFO}: scalar mediator model
\item {\tt GC\_pseudo\_UFO}: pseudoscalar mediator model.
\end{itemize}
The models are generated using FeynRules \cite{Alloul:2013bka} and use the Madgraph~5 UFO format \cite{Degrande:2011ua}. If you use these models, please cite \cite{Abdullah:2014lla} where they were originally used for studies of indirect detection with on-shell mediators.

\section{Parameters}

The dark matter particle is a Dirac fermion $\chi$ ({\tt chi}) with antiparticle $\bar{\chi}$ ({\tt chi$\sim$}). A real spin-0 particle $\phi$ ({\tt sc}) mediates interactions with the Standard Model (SM) fermions. Each model introduces 6 new parameters which can be modified in {\tt param\_card.dat}:

\begin{itemize}
\item {\bf \tt Mchi}: $m_\chi$, mass of the dark matter particle
\item {\bf \tt Msc}: $m_\phi$, mass of the scalar/pseudoscalar mediator
\item {\bf \tt gu}: $g_u$, scalar--up-type quark (up, charm, top) scaling
\item {\bf \tt gd}: $g_d$, scalar--down-type quark (down, strange, bottom) scaling
\item {\bf \tt gl}: $g_\ell$, scalar--charged lepton scaling
\item {\bf \tt gchi}: $g_\chi$, scalar--dark matter coupling
\end{itemize}

The couplings between the mediator and SM fermions is scaled by the fermion mass. Explicitly, we include the following interaction terms:

\begin{align}
\Delta\mathcal{L}_\text{scalar} 
&= 
- g_u \frac{y_u}{\sqrt{2}} \phi \bar{u}u 
- g_d \frac{y_d}{\sqrt{2}} \phi \bar{d}d 
- g_\ell \frac{y_\ell}{\sqrt{2}} \phi \bar{\ell}\ell 
- g_{\chi} \phi \bar{\chi}\chi
\label{eq:L:scalar}
\\
\Delta\mathcal{L}_\text{pseudo} 
&= 
- ig_u \frac{y_u}{\sqrt{2}} \phi \bar{u}\gamma^5 u 
- ig_d \frac{y_d}{\sqrt{2}} \phi \bar{d}\gamma^5 d 
- ig_\ell \frac{y_\ell}{\sqrt{2}} \phi \bar{\ell}\gamma^5 \ell 
- ig_{\chi} \phi \bar{\chi}\gamma^5\chi.
\label{eq:L:pseudo}
\end{align}
%

The parameters $g_u$, $g_d$, and $g_\ell$ have been chosen such that when they are set to 1, $\phi$ has the same overall couplings as the Higgs boson to Standard Model fermions. For example, when $g_u = 1$, the coupling of $\phi$ to the up quarks is $\frac{m_{u_i}}{v}$, with $v=246 \GeV$ for $u_i = u,c,t$.

The mediator interacts with gluons and photons through top quark loops. This effective coupling is included in the large top mass limit:

\begin{align}
\Delta\mathcal{L}_\text{scalar} 
&=
-\frac{1}{4} \frac{g_s^2}{4\pi} \frac{g_u}{3\pi v} t\left( \frac{m_h^2}{4m_t^2}\right) G_{\mu \nu}^a G^{\mu \nu a} \phi
-\frac{1}{4} \frac{e^2}{4\pi} \frac{8 g_u}{9 \pi v} t\left(\frac{m_h^2}{4m_t^2}\right) F_{\mu \nu} F^{\mu \nu} \phi
\label{eq:L:scalarHEFT}
\\
\Delta\mathcal{L}_\text{pseudo} 
&=
-\frac{1}{4} \frac{g_s^2}{4\pi} \frac{g_u}{2\pi v} p\left( \frac{m_h^2}{4m_t^2}\right) G_{\mu \nu}^a \tilde{G}^{\mu \nu a} \phi
-\frac{1}{4} \frac{e^2}{4\pi} \frac{4 g_u}{3\pi v} p\left( \frac{m_h^2}{4m_t^2}\right) F_{\mu \nu} \tilde{F}^{\mu \nu} \phi.
\label{eq:L:pseudoHEFT}
\end{align}
%

The functions $t$ and $p$ are based on the \texttt{HiggsEffectiveTheory} model in FeynRules \cite{HEFT}:
%

\begin{align}
t(x)&= 1 + \frac{7}{30}x + \frac{2}{21}x^2 + \frac{26}{525}x^3
\\
p(x)&= 1 + \frac{1}{3}x + \frac{8}{45}x^2 + \frac{4}{35}x^3.
\end{align}

\section{Caveats and Remarks}
\begin{enumerate}

\item The decay width of the new scalar will change as couplings to fermions are varied. Therefore, whenever couplings or masses are changed in {\tt param\_card.dat}, the width must be recalculated before events are generated. After generating a process in MG5 and producing an output folder, the new decay width of the scalar can be calculated by entering:\\

{\tt ./bin/madevent compute\_widths sc -f}\\

in the output directory of the generated process. This command will calculate the width of {\tt sc}, rewrite the {\tt param\_card.dat} with the updated decay information for the scalar, and exit madevent.

\item The mediator interactions are similar to a light Higgs boson. While gluon fusion is included in this model file, existing implementations in MadGraph for gluon fusion are not valid when the kinematics involve energy scales $E \gtrsim 2m_t$.

\item The mass-proportional couplings in (\ref{eq:L:scalar}--\ref{eq:L:pseudo}) follow from the requirement of electroweak gauge invariance which requires that the coupling goes through the Yukawa interactions. This automatically imposes minimal flavor violation. See Section~6 of~\cite{Abdullah:2014lla} for further discussion and~\cite{Ipek:2014gua} for an example of a complete theory.


\end{enumerate}



\begin{thebibliography}{90}

%\cite{Alwall:2011uj}
\bibitem{Alwall:2011uj} 
  J.~Alwall, M.~Herquet, F.~Maltoni, O.~Mattelaer and T.~Stelzer,
  %``MadGraph 5 : Going Beyond,''
  JHEP {\bf 1106}, 128 (2011)
  [arXiv:1106.0522 [hep-ph]].
  %%CITATION = ARXIV:1106.0522;%%
  %1293 citations counted in INSPIRE as of 21 Aug 2014

%\cite{Alloul:2013bka}
\bibitem{Alloul:2013bka} 
  A.~Alloul, N.~D.~Christensen, C.~Degrande, C.~Duhr and B.~Fuks,
  %``FeynRules  2.0 - A complete toolbox for tree-level phenomenology,''
  Comput.\ Phys.\ Commun.\  {\bf 185}, 2250 (2014)
  [arXiv:1310.1921 [hep-ph]].
  %%CITATION = ARXIV:1310.1921;%%
  %83 citations counted in INSPIRE as of 21 Aug 2014

%\cite{Degrande:2011ua}
\bibitem{Degrande:2011ua} 
  C.~Degrande, C.~Duhr, B.~Fuks, D.~Grellscheid, O.~Mattelaer and T.~Reiter,
  %``UFO - The Universal FeynRules Output,''
  Comput.\ Phys.\ Commun.\  {\bf 183}, 1201 (2012)
  [arXiv:1108.2040 [hep-ph]].
  %%CITATION = ARXIV:1108.2040;%%
  %132 citations counted in INSPIRE as of 21 Aug 2014

%\cite{Abdullah:2014lla}
\bibitem{Abdullah:2014lla} 
  M.~Abdullah, A.~DiFranzo, A.~Rajaraman, T.~M.~P.~Tait, P.~Tanedo and A.~M.~Wijangco,
  %``Hidden On-Shell Mediators for the Galactic Center Gamma-Ray Excess,''
  Phys.\ Rev.\ D {\bf 90}, 035004 (2014)
  [arXiv:1404.6528 [hep-ph]].
  %%CITATION = ARXIV:1404.6528;%%
  %17 citations counted in INSPIRE as of 21 Aug 2014

\bibitem{HEFT} 
  Claude Duhr,
  \texttt{http://feynrules.irmp.ucl.ac.be/wiki/HiggsEffectiveTheory}

%\cite{Ipek:2014gua}
\bibitem{Ipek:2014gua} 
  S.~Ipek, D.~McKeen and A.~E.~Nelson,
  %``A Renormalizable Model for the Galactic Center Gamma Ray Excess from Dark Matter Annihilation,''
  arXiv:1404.3716 [hep-ph].
  %%CITATION = ARXIV:1404.3716;%%
  %20 citations counted in INSPIRE as of 21 Aug 2014





\end{thebibliography}


\end{document}
